Since our code has been shipped inside a container, we needed to test if the prototype
was able to communicate with the chosen frameworks. In fact, the partition module  needs to invoke the Kubernetes API
in order to assign the community and role labels.\\
We deployed our container on Minikube. Minikube is a virtual machine running in VirtualBox, in which a
one-node Kubernetes cluester has already been deployed. The deployment of the container 
was done by using the OpenFaas CLI, and its functionalities were tested by using both the invocation
through the \textit{faas-cli} and the \textit{faas-gui}.\\
The SLPA algorithm was tested locally by creating a mock delay matrix. This matrix should be symmetric and, 
in order to have significant results, it should reflect the structure of a real network. In fact, drawing the network
delays from a uniform distribution produces a single community in which all nodes have been put but doesn't correspond
to any real-world network.
When applied to a correctly build matrix, the algorithm works as expected. The big communities broken down by our Round Robin 
implementation are still correct, but they look more sparse, since the assignment to the new smaller communities is done by
cycling on an unordered list.