The scope of this project is to develop the first of the four PAPS phases: the network 
partitioning. This part of the framework has the role of dividing the network into 
delay aware communities in order to reduce the number of nodes on which the next phases
will need to work. The first piece needed to partition the network is the network itself:
it will be provided as an input in the form of a delay matrix, from a node to any other
node, and the number of nodes which are part of that network. \\ 
Once the input is parsed and data are acquired, the program checks if the nodes specified 
in the input are really present in the cluster. To do that the program interrogates 
Kubernetes, through a call to the API, to obtain the list of objects (V1Node) related to
all the nodes registered in the cluster and checks if the names given in the
input matches with real cluster nodes. If all the matches are positive, each selected 
node is wrapped in a SLPA \cite{SLPA} node to allow an effective partition of the network.
This wrapping adds to the initial Kubernetes structure a memory and a list of nearby 
nodes and a label which it will try to spread.
%--------------- inizio descrizione SLPA -----------------------
To understand why those structures are 
added, a brief introduction on SLPA is needed. It is composed of a variable number of 
iterations (stable solutions are obtained with 20) where nodes spread their labels 
around to their neighbors. In each iteration, all the nodes will become once 
\textit{listener} and will collect a label, selected from the memory, from any nearby 
node (\textit{speaker}). Once all the labels are collected, the listener selects the 
most popular node (or selects a node with any other listening rule) and adds it to its
memory. When it will become speaker it will select a label from its memory according 
to a probability distribution based of the number of occurrences of each label in the 
memory. An iteration will end when all the nodes have been listener once. \\
%----- fine circa
After the given number of iterations, the memory of each node will be post-processed, 
the most popular node in the memory will determine its community. In case of multiple
labels with the same number of occurrences, the node will be assigned to multiple 
communities and will be considered an \textit{overlapping node}. Since PAPS doesn't 
want to deal with those kind of nodes, the overlap will be solved selecting randomly 
one of the overlapping communities. \\
With this procedure the network will be effectively divided into communities but, 
to keep the size even more under control, oversized communities will be split to better
fit the size limit (also provided as input). \\
Now that all the nodes are assigned to a community, another API call will modify the
existing node that will add a label in order to easily recognize and manage each community.

