The scope of this project is to develop the first of the four PAPS phases: the network 
partitioning. This part of the framework has the role of dividing the network into 
delay aware communities in order to reduce the number of nodes on which the next phases
will need to work. The first piece needed to partition the network is the network itself:
it will be provided as an input in the form of a delay matrix, from a node to any other
node, and the number of nodes which are part of that network. \\ 
Once the input is parsed and data are acquired, the program checks if the nodes specified 
in the input are really present in the cluster. To do that the program interrogates 
Kubernetes, through a call to the API, to obtain the list of objects (V1Node) related to
all the nodes registered in the cluster and checks if the names given in the
input matches with real cluster nodes. If all the matches are positive, each selected 
node is wrapped in a SLPA node to allow an effective partition of the network.
This wrapping adds to the initial Kubernetes structure a memory and a list of nearby 
nodes and a label which it will try to spread.\\
To run SLPA, the program takes as inputs, the number of iterations, a probability threshold 
and a maximum size for the communities. To improve the randomness of the label spreading, 
both the list of the listeners and the list of the speakers are shuffled at each iteration.
For each listener in the topology, its neighbors will become the speakers and from them it
will obtain a list of labels and the most received one will be added to the memory 
(\textit{listening rule}). Every time the memory is updated, the node will select the 
label that he will spread according to the \textit{speaking rule}. In our implementation
the speaking rule uses a \textit{ranking selection} algorithm which randomly picks a node
according to a probability distribution based on the occurrences of each label
in the memory. If a node is present many times in the memory it will be select with an 
higher probability. \\
After the given number of iterations, the memory of each node will be post-processed, 
the most popular node in the memory will determine its community. In case of multiple
labels with the same number of occurrences, the node will be assigned to multiple 
communities and will be considered an \textit{overlapping node}. Since PAPS doesn't 
want to deal with those kind of nodes, the overlap will be solved selecting randomly 
one of the overlapping communities. \\
With this procedure the network will be effectively divided into communities but, 
to keep the size even more under control, oversized communities will be split to better
fit the size limit (also provided as input). \\
Now that all the nodes are assigned to a community, another API call will modify the
existing node that will add a label in order to easily recognize and manage each community.

