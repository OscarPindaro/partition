The scope of this project is to develop the first of the four PAPS phases: the network 
partitioning. This part of the framework has the role of dividing the network into 
delay aware communities in order to reduce the number of nodes on which the next phases
will need to work. The first pice needed to partition the network is the network itself,
this will be provided as an input in the form on a delay matrix from a node to any other
node and the number of nodes forming the network. \\ 
Once the input is parsed and data are acquired, the program checks if the nodes specified 
in the matrix are really present in the cluster. To do that the program interrogates 
the cluster, through a call to the API, to obtain the list of kubernetes objects (V1Node)
which contains all the node registered in the cluster and checks if the name given in the
input file corresponds to a real node. If all the matches are positive, each selected 
node is wrapped in a SLPA \cite{SLPA} node to allow an effective partition of the network.
This wrapping adds a to the initial kubernetes structure a memory and a list of nearby 
nodes. To understand why those structures are added to each node, a brief introduction on
SLPA is needed, this algorithm is used to divide a network of interconnected elements into
communities spreading labels in the network to decide to which community each node belongs to.


--------------------------------------------------------------------

he partitioning is mainly focussed on the division of the network in multiple 
    communities in order to reduce the scope of the following parts. The division is performed using 
    the SLPA \cite{SLPA} algorithm, which uses a label spreading technique to assign each node to a
    community. Once all the nodes are assigned, the Kubernetes node will be modified using the API 
    labeling it in order to easily recognize and manage each community.