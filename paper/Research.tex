\subsubsection*{Framework research}
To start the process of developing PAPS, the first thing was to search for any useful tool 
that could speed up the process of creating what was designed.  
The starting idea was to use OpenWhisk \cite{OpenWhisk}, a serverless platform able to run 
functions on top of an infrastructure. The idea was to integrate it on top of a kubernetes 
cluster, running a series of docker \cite{Docker} containers. \\
Searching on the internet information about the integration of those platforms, other two 
option were available: OpenFaas \cite{Faas} and Kubeless \cite{Kubeless}. For what concerns 
the functionalities the three frameworks are pretty similar one to another. OpenWhisk and 
OpenFaas work almost in the same way, both are used in many applications nowadays. 
Kubeless has a built in synergy with kubernetes, since is written on top of it,  and for 
that reason could be a good alternative. The main issue with OpenWhisk and Kubeless is 
that they are written in Scala and Akka and their structure is pretty complex so they could
result complex to use and understand them.
