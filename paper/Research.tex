\subsubsection*{Framework research}
To start the process of developing PAPS, the first thing was to search for any useful tool 
that could speed up the process of creating what was designed.  
The starting idea was to use OpenWhisk \cite{OpenWhisk}, a serverless platform able to run 
functions on top of an infrastructure. The idea was to integrate it on top of a kubernetes 
cluster, running a series of docker \cite{Docker} containers. \\
Searching on the internet information about the integration of those platforms, other two 
option were available: OpenFaas \cite{Faas} and Kubeless \cite{Kubeless}. For what concerns 
the functionalities the three frameworks are pretty similar one to another. OpenWhisk and 
OpenFaas work almost in the same way, both are used in many applications nowadays. 
Kubeless has a built in synergy with kubernetes, since is written on top of it,  and for 
that reason could be a good alternative. The main issue with OpenWhisk and Kubeless is 
that they are written in Scala and Akka, OpenWhisk also has a complex structure and for that reason 
is not optimal for a project like this. For what concerns OpenFaas, it is written
in Go and offers a lot of additional features like a metrics system and the possibility to 
customize the scheduling policy used by the framework. During the research process one main
problem came out, all the frameworks have the characteristic to hide to the final user 
the underlying structure making it hard to integrate PAPS functionalities in it.
To solve this issue, the best decision was to build something directly on top of kubernetes
using the provided API, available for a large variety of languages, to create, manage and 
delete containers when needed. \\
For what concerns the framework, OpenFaas cam in handy to provide a frontend module 
useful for PAPS users which will be easily able to upload the functions that PAPS
will manage and distribute. OpenFaas was the chosen one since it offers both a
simple UI and the possibility to be managed by a cli. It also offers the possibility to 
install a statistics collection module which will be really useful to monitor how the
system is working in a real environment.

\subsubsection*{Kubernetes integration}