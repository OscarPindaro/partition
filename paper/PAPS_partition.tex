\documentclass{article} %document style and layout
\usepackage[table, dvipsnames]{xcolor} %colors for tables and text
\usepackage{graphicx} %
\usepackage{tabu} %creation of tables
\usepackage{array}
\usepackage{enumitem} %list of elements
\usepackage{hyperref} %insert clickable links
\urlstyle{same}
\usepackage{subfig} 
\setcounter{lofdepth}{2} %subfig used to have more than 1 figure together, \setcounter used to add subfigures in the list of figures
\usepackage{float}

%-------Titlepage information------------------------------------------
\title{\textbf{\huge{PAPS: A first step into the implementation phase. Partitioning a network.}}}
\author{\color{black}Fabio Losavio, Oscar Francesco Pindaro \\ \\
        Dipartimento di Elettronica, Informazione e Bioingengeria \\
        Politecnico di Milano, Italy \\
        \texttt{\{fabio.losavio,oscarfrancesco.pindaro\}@mail.polimi.it}}
\date{xx - May - 2020}
%------------------------------------------------------------------------


%DOCUMENT BEGINNING
\begin{document}
%TITLEPAGE and ABSTRACT
\maketitle

\begin{abstract}
    The emergence of latency-sensitive and data-intensive applications requires to move the 
    computational power closer to users on nodes at the edge of the network (\textit{edge computing}).
    This work starts from PAPS \cite{PAPS}, a framework which aims to tackle the complexity of edge 
    infrastructures by means of decentralized self-management and serverless computing. This paper
    shows how the first of the four PAPS phases, \textit{partitioning}, is implemented and integrated
    with a common open-source system: Kubernetes \cite{Kube}, to easily deploy the system in an already
    existing network. The partitioning is mainly focussed on the division of the network in multiple 
    communities in order to reduce the scope of the following parts. The division is performed using 
    the SLPA \cite{SLPA} algorithm, which uses a label spreading technique to assign each node to a
    community. Once all the nodes are assigned, the Kubernetes node will be modified using the API 
    labeling it in order to easily recognize and manage each community.
    \\ \\
    \textbf{Keywords:} Edge computing \textbf{\textperiodcentered} Partition 
    \textbf{\textperiodcentered} Kubernetes \textbf{\textperiodcentered} PAPS
    \textbf{\textperiodcentered} Community division \textbf{\textperiodcentered} SLPA
    
\end{abstract}
%------------------------------------------------------------------------------------------------------------------------------------------------
{{\section{Introduction}\label{sect:intro}}}
PAPS is an edge computing framework which aims to improve network performances of some 
applications moving the computation closer to the final user instead of doing that on 
a centralized server. This will reduce the amount of traffic flowing into the network
with the result of an improvement of the throughput allowing to better meet the SLAs
of delay sensitive applications.

This framework focuses on MEC topologies, composed of geo-distributed nodes which access
the system through cellular base stations. In a typical topology is possible to identify
two different networks: the \textit{fronthaul network}, which connects normal nodes to 
the MEC stations, and the \textit{backhaul network}, which interconnects the MEC stations.
PAPS aims to reduce the complexity of the problem working at three different levels: 
\textit{system, community and node}. 
\\
\subsubsection*{System level}
Since MEC topologies can be really big and complex, the first step is to \textit{partition}
the network in multiple delay-aware sub-networks called \textit{communities}. Each community
is composed by a set of nodes whose propagation delay from one another is below a given 
threshold. PAPS assumes the availability of a \textit{supervisor} that has a global view of
the MEC topology and uses SLPA \cite{SLPA} to create the communities. Each community will
elect a \textit{leader} which will be in charge of managing the community and the communication
between the nodes.

\subsubsection*{Community level}
Communities aims to minimize the likelihood of of SLA violations so that the MEC nodes
can operate under feasible conditions.\\
Each community leader will manage the \textit{allocation and placement} phases by looking at 
the aggregate demand to each service and the node capacities in order to decide how to 
distribute resources among the nodes in its community. This is done solving a \textit{mixed
integer programming (MIP)} problem whose goal is to minimize the overall community delay. 
The MIP is solved periodically by the community leader which will then contact the other members
to communicate the new configuration.

\subsubsection*{Node level}
The Node-Level Self-Management system aims to keep the response time for a given function fixed, 
by deploying and removing containers accordingly to the measured workload.
In order to achieve this, every node has a controller for each function that is hosting; all the controllers
run in parallel and independently.
Each controller accepts as input the number of container allocated for its function and the 
measured arrival rate, and outputs the response time. The objective is to keep the response time
at the control set point, which must be lower than the SLA.





%------------------------------------------------------------------------------------------------------------------------------------------------
\clearpage
{{\section{References}\label{sect:biblio}}}
\begin{thebibliography}{9}
    \bibitem{PAPS}
        Baresi, L., Mendonça, D., Quattrocchi, G.: 
        \textit{PAPS: A Framework for Decentralized Self-management at the Edge.}
        In: Service-Oriented Computing, pp.508-522, (2019).   

    \bibitem{SLPA}
        Xie, J., Szymanski, B K, Liu, X.:
        \textit{SLPA: Uncovering Overlapping Communities in Social Networks via
        A Speaker-listener Interaction Dynamic Process.}
        In Proc: Data Mining Technologies for Computational Collective 
        Intelligence Workshop at ICDM, Vancouver, CA pp. 344-349, (2011)

    \bibitem{Edge}
        Shufen Wang:
        \textit{Edge Computing: Applications, State-of-the-Art and Challenges, Advances in Networks.}
        Vol. 7, No. 1, pp. 8-15. doi: 10.11648/j.net.20190701.12 (2019)

    \bibitem{Kube}
        https://kubernetes.io/

    \bibitem{Docker}
        https://www.docker.com/

    \bibitem{OpenWhisk}
        https://openwhisk.apache.org/

    \bibitem{Faas}
        https://www.openfaas.com/

    \bibitem{Kubeless}
        https://kubeless.io/

    \bibitem{GANXis}
        https://sites.google.com/site/communitydetectionslpa/    

\end{thebibliography}



\end{document}