PAPS is an edge computing framework which aims to improve network performances of some 
applications moving the computation closer to the final user instead of doing that on 
a centralized server. This will reduce the amount of traffic flowing into the network
with the result of an improvement of the throughput allowing to better meet the SLAs
of delay sensitive applications.

This framework focuses on MEC topologies, composed of geo-distributed nodes which access
the system through cellular base stations. In a typical topology is possible to identify
two different networks: the \textit{fronthaul network}, which connects normal nodes to 
the MEC stations, and the \textit{backhaul network}, which interconnects the MEC stations.
PAPS aims to reduce the complexity of the problem working at three different levels: 
\textit{system, community and node}. 
\\
\subsubsection*{System level}
Since MEC topologies can be really big and complex, the first step is to \textit{partition}
the network in multiple delay-aware sub-networks called \textit{communities}. Each community
is composed by a set of nodes whose propagation delay from one another is below a given 
threshold. PAPS assumes the availability of a \textit{supervisor} that has a global view of
the MEC topology and uses SLPA \cite{SLPA} to create the communities. Each community will
elect a \textit{leader} which will be in charge of managing the community and the communication
between the nodes.

\subsubsection*{Community level}
Communities aims to minimize the likelihood of of SLA violations so that the MEC nodes
can operate under feasible conditions.\\
Each community leader will manage the \textit{allocation and placement} phases by looking at 
the aggregate demand to each service and the node capacities in order to decide how to 
distribute resources among the nodes in its community. This is done solving a \textit{mixed
integer programming (MIP)} problem whose goal is to minimize the overall community delay. 
The MIP is solved periodically by the community leader which will then contact the other members
to communicate the new configuration.