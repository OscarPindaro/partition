Thanks to the results of the research phase, we decided to use mainly kubernetes functionalities
and extension possibilities to implement PAPS phases and use OpenFaas as a frontend module to 
allow an easy deployment of the serverless functions.

The first of the four PAPS phases is the \textit{partition}, in this phase the original 
network topology on which PAPS will need to work is divided into delay-aware communities in 
order to reduce the scope of the following parts. To do that the idea is to take as input a 
delay matrix describing the measured delay between the nodes of the networks and other 
additional parameters that can be useful to describe the desired behavior of the partitioner.
This input can be provided as a file input or even through the OpenFaas GUI and will than be
parsed by the program to fill some data structures. The partition will first of connect to 
kubernetes API manager and do a call to obtain the list of registered nodes to check if they 
match with what has been parsed from the input. This operation needs to be done in order to
avoid the presence of nodes in the topology that are not actually managed by kubernetes and 
so that can't be modified and labeled according to the specific needs. 

To divide the network the partition will use SLPA, an algorithm which uses label spreading 
techniques to define some  