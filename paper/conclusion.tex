Our work focused mostly on finding the best frameworks that would support PAPS functionalities.
These serverless computing frameworks turned out to be excellent tools for the production
and deployment of stateless functions, but they also had evident limits in managing the underlying
infrastructure.
Since we needed a much lower level of abstraction, we decided to use also Kubernetes and extend
its functionalities, taking advantage of its modular architecture.
We implemented the Partition module on Kubernetes, and we think that the remaining modules will
be placed here as well. In this way, OpenFaas has the role of a frontend module, that allows
the user to upload its function in the cluster without having to care about the infrastructure.
We chose OpenFaas for its excellent support and simplicity, but it can be substituted with other frameworks
with very little effort, since PAPS works at the Kubernetes level.
PAPS is mostly focused on a workload that come from outside the MEC network, and so the trigger functionality,
that links the execution of multiple functions, was very marginal.
If a future work will focus on a micro-services architecture, OpenWhisk may be
a better option, since it offers a cleaner and more powerful structure for this mechanism.
\par
For what concerns the implementation of the Partition module, we decided to write ourselves 
the SLPA algorithm, and we used a round robin approach to break down too large communities.
This part of the component could be substituted by using the GANXiS \cite{GANXis} software, a program
that runs the same algorithm and has more complex and effective techniques to reduce
the size of the communities inside the given boundaries.
The next step in the development of PAPS will have to be the implementation of the Allocation 
and Placement components. Their functioning is completely decoupled from the Partition module,
since they just need to know the structure of the communities, and any change in its logic
will have no effect on how the container will be disposed inside the cluster.
Kubernetes offers very powerful tools that allows to schedule and place pods on the nodes of the cluster,
while taking care that every container is healty and running. Thanks to this functionality,
the Allocation and Placement will just have to produce an allocation plan without having to
care how to enforce it.